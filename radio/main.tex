\documentclass[11pt,a4paper,headinclude=true,DIV=14,BCOR=8mm,chapterprefix,listof=totoc,twoside,openright,abstracton]{scrbook}

\usepackage[headsepline]{scrpage2}
\usepackage[utf8]{inputenc}
\usepackage{geometry}
\usepackage{amssymb}
\usepackage{amsthm}
\usepackage{enumerate}
\usepackage{graphicx}
\usepackage{float}
\usepackage[intlimits]{amsmath}
% \usepackage{siunitx}
% \usepackage{color}
\usepackage{xcolor}
\usepackage{verbatim}
\usepackage{appendix}
\usepackage{hyperref}
\usepackage{hyperref}
\usepackage{mathtools}
% \usepackage[style=authoryear]{biblatex}
\usepackage{natbib}
% \usepackage{newtxtext}
% \usepackage{newtxmath}
% \usepackage{harvard}
\setcitestyle{aysep={}} 
\bibliographystyle{apalike}
\usepackage{xr}
\usepackage{wrapfig}
% \bibliographystyle{agsm}
%\usepackage{feynmf}
%\usepackage{tensor}
\usepackage[framemethod=tikz]{mdframed} % for a block of text

\setlength{\parindent}{0pt}
\geometry{a4paper, tmargin=3cm, bmargin=3cm, lmargin=3cm, rmargin=3cm, headheight=3em, headsep=2em, footskip=1cm}

\setcitestyle{citesep={,}}

\newcommand{\todo}[1]{\textcolor{red}{$\blacksquare$ TODO: #1}} 

\newmdenv[linecolor=cyan,backgroundcolor=cyan!20]{sidenote}


\geometry{a4paper, tmargin=2cm, bmargin=2cm, lmargin=1cm, rmargin=1cm, headheight=2em, headsep=2em, footskip=1cm}

\title{PhD thesis}
\author{Vsevolod Nedora}
\date{today}

\begin{document}

\section{Synchrotron Spectra of Non-uniform compact sources}
\textit{Gordon and Breach, 1973}


Two classes of extragalactic radio-sources can be selected. The firs has $\nu\approx 1$GGz, comtains sources with spectral index $\alpha=-0.75$, that are constant of slowly decrease with $\nu$. \\
\textbf{Fist class} has the stright spectra -- are the sources that are optically thin with power-law distribution of electron energies $N(E) = N_0 E^{\gamma}$ where $\alpha = (1-\gamma/2)$. Concave downward $C^{-}$ spectra are evolutionary consequence of different energy dependences of electron power losses.\\
\textbf{Second class} has a $\alpha>-0.5$. In particular, those with $\alpha>0.33$ are commplex sources with self-absorption. Their number, however is small. Majority of sources have intermediate spectra with $\alpha$ slowly decreasing with $\nu$. Their exact origin is difficult to understand, as such spectra require either very low magnetic field energies with high electron energies or, if a combined spectra is concerned, the combination that yield $\alpha=0$ is unlikely. \\

In addition, observed "stright" spectra becomes complex when hore observations is done. \\

A sharp spectral peak characterisitc is a indication of a self-absorbing source. \\

A minima in the spcectra can be a result of a combined spectra which component peaks are separated in frequency.


\subsection{Model}

Consider a source whose structure is described by the radial dependence of magnetic field and particle density. Thus we consider a spherically symmetrical source with 

\begin{equation}
    r^2 = \sqrt{x^2 + y^2 + z^2},
\end{equation}

\textcolor{gray}{where however a different geometry was intiially applied $r = |x| + (y^2 + z^2)^{1/2}$ -- double cone.}




\subsection{Artuch \& Nedora}

Let us first consider the physical parameters of a radio sources. In case of a uniform source the magnetic inductivity $B_{\perp}$ of a unifrom radio source is determind from observations at a low $\nu_{LF}$ in the optically thick region 

\begin{equation}
    B_{\perp} = b^2(\gamma)\frac{(\nu_{LF}/2C_1)^5 (\Omega_{LF})^2}{S_{LF}^{2}}\frac{\delta}{1+z}
    \label{eq:art:bperp}
\end{equation}

where $B_{\perp} = B\sim\theta$, where $\theta$ is an angle between the direction of the magnetic field and the line of site, $\Omega_{LF}$ is the solid angle of the radio source, which observed at low frequency (in turn, the linear angular size of the source $\theta_{LF} = \sqrt{\Omega_{LF}}$), $S_{LF}$ its flux at the frequency $\nu_{LF}$, and $\delta$ is the Doppler factor of the radio source, $z$ is the red shift of the parent galaxy, $\gamma$ is the exponent i nthe spectrum $N(E) = N_0E^{-\gamma}$, $b(\gamma) = c_5(\gamma)/c_6(\gamma)$ are the functions tabulated, similarly to the constant $C_1$. \\

Using the estimate fro $B_{\perp}$ from observations at high frequency $\nu_{HF}$ (in the optically thin region), we obtain an estimate for the coefficient $N_0$

\begin{equation}
    N_0 = \frac{1}{c_5(\gamma)}\frac{S_{HF}}{\Omega_{HF}\theta_{LF}D_{PH}}B^{-(\gamma+1)/2}\Big(\frac{\nu_{HF}}{2C_1}\Big)^{(\gamma-1)/2}\frac{(1+z)^{(9+\gamma)/2}}{\delta^{(5+\gamma)/2}}
    \label{eq:art:n0}
\end{equation}

where $D_{PH}$ is the photometric distance to the parent galaxy. In equations \ref{eq:art:bperp} and \ref{eq:art:n0}, angular sizes of the source at high and low frequencies appear. \\
Notably, that only a source if a form of a cylinder with an axis directed along the line of site, would have a visible angular size that is independent of frequency and equal to the physical size of the source. For other geometries of a uniform sources and for non-uniform sources the angular size depends on the frequency. \\

Thus the radio source angular size is model-dependent. A model must be physically motivated to the maximum degree. Here we consider a spherical model for a non-uniform radio source, where the magnetic fields and particle number densities are distributed accordong to a power law:

\begin{align}
    B_{\perp} &= \frac{B_{\perp}(0)}{1 + k_H(r/R)^m},\\
    N(r) &= \frac{N(0)}{1 + k_N(r/R)^n}
\end{align}

where $B_{\perp}$ is the component of the magnetic induction perpendicular to the line of sight, $R$ is the source radius, $k_H$ is the non-uniformity coefficient for the magnetic field, and $k_N$ is the non-uniformoty coefficient for the particle distribution. \\

Our models shows that the non-uniformity of particle distribution can be neglected. For the magnetic field de Bruyn \url{https://ui.adsabs.harvard.edu/abs/1976A\%26A....52..439D/abstract} have shown that in a semi-transparent region, the spectral index is given 

\begin{equation}
    \alpha_{LF} = \frac{13 - 5n - 3m - 2m\gamma + 2\gamma}{2 - 2n - 2m - \gamma}
\end{equation}

For a uniform spatial distribution of particles $n=0$ and $h_N = 0$. Extragalactic radio sources in semi-transparent regime have flat spectra, \textit{i.e.,} average spectral index $\langle\alpha\rangle\approx0.9$, then $\langle\gamma\rangle = 2\alpha+1 = 2.8$, result in $m\approx 2$. \\

We further assume that the $B(0) = 0.1$ Gauss, source radius $R=1$pc and the distance to the source is $D=50$Mpc. Thes parameters correspond to the source with a peak of $1$Jy around $1$GHz. And while the parameters of the real source would of course deviate from assumed here, the $\tau=1$ at the spectra maximum. At higher frequencies the source is transparent, while at lower it is opaque. \\

We have shown that the dependence of the angular size on the spectrum does not however depend on the prescribed parameters, such as $B(0)$, $R$, $D$. \\

The size of the source is determined at half maximum emission power relative to the full size of the source $\rho_{0.5}(\nu)/R$, which depends on the emission frequency. While fixing $\gamma=2.8$, we consider different degrees of MF non-uniformity $k_H\in{0,10^5}$. At low frequencies (optically thick regime) we see only the emission from a thin layer of the soruce surface, facing the observer. Thus, we see an actual physical size of the source $R$, that corresponds to the visible angular size $\theta_{LF}$. Hence $\rho_{0.5}(\nu)/R = \theta(\nu)/\theta_{LF}$, where $\theta(\nu)$ is the visible angular size of the source at half power. We compute $\theta(\nu)$ for models with various $k_H$. \\

Figure shows, that at low frequencies we observe the actual size of the source. As we increase the $\nu$, the $\theta$ of the soruces decreases exponentially. For a uniform object this decrease if small by $\sim 13\%$ ($k_H=0$), the change in $\theta$ becomes considerable is we increase the degree on non-uniformity of our model $k_H$. 

Investigation of a large number of models led to the conclusion that the $\theta_{LF}/\theta_{HF}$ depends primarily on $k_H$. 

\subsubsection{Estimation of the angular size of a source at low frequencies}

Let us consider the procedure of estimating the $\theta_{LF}$. We start by considering the optically thin regime, where the source is transparent (high frequency)

We start by introducing constants from Paholchic. Let $e$ be the electron charge, $m_e$ the electron mass, $c$ the speed of light,

\begin{align}
    c_1 &= \frac{3 e}{4 \pi m_e ^3 c^5} \\
    a &= \Gamma\Big(\frac{3 \gamma - 1}{12}\Big) \Gamma\Big(\frac{3\gamma + 7}{12}\Big) \\
    c_5 &= \frac{\sqrt{3}e^3}{16\pi m_e c^2}\frac{\gamma + (7/3)}{\gamma + 1} a \\
    b &= \Gamma\Big(\frac{3\gamma + 2}{12}\Big)\Gamma\Big(\frac{3\gamma + 10}{12}\Big) \\
    c_6 &= \sqrt{3}\pi e m_e ^5 c^{10} (\gamma + 10/3) (b / 72) 
\end{align}

where $\Gamma(\cdot)$ is the gamma function. \\
then for simplicity we introduce following quntities that depend only on basic model parameters

\begin{align}
    \epsilon_0 &= c_5 H_{0}^{(\gamma + 1)/2} \Big(\frac{\text{GGz}}{2 c_1}\Big)^{(1-\gamma)/2} \\
    \mu_0 &= c_6 H_0^{(\gamma + 2)/2} \Big(\frac{\text{GGz}}{2 c_1}\Big)^{-(\gamma+4)/2}
\end{align}

where GGz is $10^9$. To express 









%% --------------- 
%%
%% References
%%
%% ---------------

\newpage

\bibliography{references}

\end{document}
